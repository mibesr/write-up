\documentclass{beamer}

\mode<presentation>{
  \usetheme{default}
  \usefonttheme{serif}
  \setbeamercovered{transparent}
  \setbeamerfont{institute}{size=\normalsize}
  \setbeamerfont{section in toc}{size=\large}
  \setbeamerfont{subsection in toc}{size=\normalsize}
}

\usepackage[english]{babel}
\usepackage{times}
\usepackage{algorithmic}
\usepackage{algorithm}
\usepackage{epsfig}
\usepackage{pgf}
\usepackage[absolute,overlay]{textpos}
%--------------------------------------------------
% \usepackage{mathtime}
%-------------------------------------------------- 

% Latin-1 only
\usepackage[latin1]{inputenc}
\usepackage[T1]{fontenc}
%--------------------------------------------------
% % Chinese-support
% \usepackage[nocjkbg5]{ucs}
% \usepackage[utf8x]{inputenc}
% \usepackage[C00,T1]{fontenc}
% \newcommand\tradtext[1]{\bgroup\fontencoding{C00}\fontfamily{ming}\selectfont%
% \SetUnicodeOption{cjkbg5}#1\egroup}
%-------------------------------------------------- 

\title[]{Multiple-Representation Relevance Model}
\author[]{Ruey-Cheng Chen <cobain@turing.csie.ntu.edu.tw>}
\institute{National Taiwan University, Taiwan}
\date[]{\today}
\subject{Information Retrieval}
%--------------------------------------------------
% \AtBeginSubsection[]
% {
%   \begin{frame}<beamer>
%     \frametitle{Outline}
%     \tableofcontents[currentsection,currentsubsection]
%   \end{frame}
% }
%-------------------------------------------------- 
%\beamerdefaultoverlayspecification{<+->}
\newcommand<>\ul[2]{\textbf{#1}\begin{itemize}#2\end{itemize} \vskip 0.5em}
\newcommand<>\ull[1]{\begin{itemize}#1\end{itemize}}
\newcommand<>\ol[2]{\textbf{#1}\begin{enumerate}#2\end{enumerate}}
\newcommand<>\oll[1]{\begin{enumerate}#1\end{enumerate}}
\newcommand<>\dl[2]{\textbf{#1}\begin{description}#2\end{description}}
\newcommand<>\f[2]{\begin{frame}{#1}#2\end{frame}}
\newcommand<>\mydef[2][]{\textbf{#1}\par#2\par}
\newcommand<>\tbl[3]{\textbf{#1}\vskip 0.5em\begin{center}\begin{tabular}{#2}#3\end{tabular}\end{center}}
\newcommand<>\fig[3]{\textbf{#1}\vskip 0.5em\begin{center}\begin{figure}\includegraphics[#2]{#3}\end{figure}\end{center}}
\newcommand<>\cols[1]{\begin{columns}#1\end{columns}}
\newcommand<>\col[2]{\begin{column}{#1}#2\end{column}}

\begin{document}

\frame{ \titlepage }

%--------------------------------------------------
% \f{References}{
%   \ul{Mandatory References}{ 
%     \item Craswell, N. and Szummer, M. 2007. Random walks on the click graph.
%     In \emph{Proceedings of the 30th Annual international ACM SIGIR Conference
%     on Research and Development in information Retrieval} (Amsterdam, The
%     Netherlands, July 23 - 27, 2007). SIGIR '07. ACM, New York, NY, 239-246.
%   }
% 
%   \ul{Authors}{ 
%     \item Nick Craswell: \footnotesize{\url{ http://research.microsoft.com/~nickcr/ }}
%     \item Martin Szummer: \footnotesize{\url{ http://research.microsoft.com/~szummer/ }}
%   }
% }
%-------------------------------------------------- 

\f{Outline}{ \tableofcontents }

\section{Introduction}
\f{Document Representation}{
  \ul{In conventional thinking...}{
    \item A document is represented as only a collection of term
    \item Every document possesses only \textbf{one} such representation
    \item This leads to simple, effective retrieval models
  }

  \ul{What if we also have these data associated with documents?}{
    \item Annotations
    \item Metadata
    \item Tags
  }
}

\section{Model}
\subsection{The Generative Framework}
\subsection{Inference}
\subsection{Hyperparameters and the Index Structure}

\section{Experimental Results}
\subsection{Query Refinement Using the Secondary Representation}
\subsection{Retrieving Query-Relevant Facets}

\section{Related Work}

\section{Discussion and Concluding Remarks}
\f{Discussion}{
}
%--------------------------------------------------
% \section{Background}
% \subsection{Graph-Theoretical Concepts}
% \f{Graph}{
%   \ul{Basic Concepts}{
%     \item An undirected graph $G = (V, E)$ is a collection of nodes $V$ and edges $E$.
%     \item Two nodes $i, j \in V$ are \emph{neighbors} if $(i, j) \in E$.
%     \item Neighborhood of a node $i$: $\{j \mid (i, j) \in E\}$.
%     \item A node is not a neighbor of itself, i.e., $(i, i) \notin E$.
%     \item A clique of $G$ is either a single node or a complete subgraph of $G$.
%   }
%   \begin{figure}
%     \centering
%     \includegraphics[width=100pt]{clique}
%   \end{figure}
% }
% 
% \subsection{Random Fields}
% \f{Random Fields}{
%   \mydef[Random Fields]{
%     Let $F = \{X_1, \ldots, X_m\}$ be a set of random variables defined on the
%     sample space $\Omega$, in which each $X_i$ takes a value $x_i \in
%     \mathcal{L}$.  A probability measure $p$ is a \emph{random field} if
%     $p(\omega) > 0$ for all $\omega \in \Omega$.
%   }
%   \vskip1em
%   \ul{Random Fields on $G$}{
%     \item Assign each random variable $X_i$ to a node of $G$.
%     \item Values of random variables are usually spatially correlated.
%   }
%   \begin{figure}
%     \centering
%     \includegraphics[width=100pt]{rf}
%   \end{figure}
% }
%-------------------------------------------------- 
\section<presentation>*{\appendixname}

\begin{frame}%[allowframebreaks]
  \frametitle{References}
  \small
  \bibliography{report}
  \bibliographystyle{alpha}
\end{frame}

\frame{
  \begin{center}
    \Huge{Thanks for your attention!}
    \par
    Any Question?
  \end{center}
}

\end{document}
